\documentclass[]{article}

%opening
\title{}
\author{}

\begin{document}
\maketitle
\section{Clase}
trucos $3/2$ $\rightarrow$ dejarlo asi, si\part{title} lo escribo como $1.50000$ pierde precisión.


Standard $IEEE754$ permite obtener mismos resultados y mismos errores.


Notación de punto flotante:
Escribir algo de manera científica con potencia 2, (insertar definición)
\begin{equation}
x_{float}=(-1)^{s}xmantissax2^{expfld-bias}
\end{equation}
Para una representación de 32 bits, 1 para el signo, 8 para el exponente quedan 23 para la mantissa.


Precisión $10^{-7}$ mas bajo de ese rango, no se pueden distinguir los números.

Densidad de números entre -1 y 1 es el $50\%$ de los números totales.

Overflow $\rightarrow$ intentar representar números mas alla de la capacidad.
Underflow $\rightarrow$ Intentar representar números demasiado pequeño, lo trunca a cero.
Truncamiento el número a representar no esta definido y el computador lo representa en el siguiente número posible.

Objetivo disminuir truncamiento.

En el computador no se preservana lgunas propiedades matemáticas:
\begin{enumerate}
	\item Asociatividad
	\item Conmutatividad
\end{enumerate}

\end{document}
